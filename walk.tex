\documentclass{amsart}
\usepackage{amsthm}
\usepackage{enumerate}
\usepackage{xcolor}

\theoremstyle{definition}
\newtheorem{theorem}{Theorem}
\newtheorem{problem}{Problem}
\newtheorem{corollary}{Corollary}
\newtheorem{remark}{Remark}
\newtheorem{conjecture}{Conjecture}
\newtheorem{lemma}{Lemma}
\newtheorem{step}{Step}
\newtheorem{case}{Case}
\newtheorem{definition}{Definition}
\newcommand{\ncc}{c}

\begin{document}
\author{Gao Mou}
\address{School of Physical and Mathematical Sciences, Nanyang Technological University, Singapore} 
\email{gaom0002@e.ntu.edu.sg} 
\author{Dmitrii V. Pasechnik}
\address{Department of Computer Science, The University of Oxford, UK}
\email{dimpase@cs.ox.ac.uk}

\title{On $k$-walks in $2K_2$-free graphs}
\begin{abstract}
We show that for any $\epsilon>0$ every $(\frac{1}{k-1}+\epsilon)$-tough
$2K_2$-free graph admits a $k$-walk, and it can be found in polynomial time. 
For this class of graphs, this proves a
long-standing conjecture due to Jackson and Wormald (1990). The proof uses a
decomposition of a $2K_2$-free graph into a coclique and an ``almost Hamiltonian''
induced subgraph, which might be of independent interest.

(ADDENDUM): Possibly using Veldman's results from \cite{veldman83}, this can be improved so that
one can always take $\epsilon=0$.
\end{abstract}

\maketitle

\section{Introduction}
A graph $G$ is called $\beta$-{\em tough}, for a real $\beta>0$, if for any $p\geq 2$ it
cannot be split into $p$ components by removing less than $p\beta$ vertices.  
This concept, a measure of graph connectivity and ``resilience'' under vertex subsets removal,
was introduced in 1973 by V. Chv\'{a}tal 
while studying   Hamiltonicity of graphs. For a survey of results on graph toughness till 2006
cf. \cite{MR2221006}.

Let $p*G$ denote the multigraph obtained from $G$ by taking each edge $p$ times. 
A $k$-{\em walk} is a spanning subgraph $W$ of $(2k)*G$, such that each vertex of $W$ 
has even degree at most $2k$. % (the evenness says that $W$ is Eulerian). 
In particular a graph has a 1-walk if and only if it is $K_2$ (i.e. one edge) or Hamiltonian.
In 1990 B.~Jackson and N.~Wormald conjectured \cite{jackson1990k} that for any integer $k\ge2$ a
$\frac{1}{k-1}$-tough graph $G$ admits a $k$-walk.
{For a survey of results on walks in graphs till 2005 cf. \cite{kouider2005connected}.}

In this paper, we prove that this conjecture is ``almost'' true, under the
assumption that $G$ is  $2K_2$-free, that is $G$ does not contain an induced
copy of the disjoint union of two edges.  Here by ``almost'' we
mean that we need to replace $\frac{1}{k-1}$ by
$\frac{1}{k-1}+\epsilon$, with $\epsilon>0$ arbitrarily small.  In fact
$\epsilon$ can be set to 0 for most $2K_2$-free graphs, and the outcome of
Theorem~\ref{thm2} will remain true, cf. Theorem~\ref{opt} below for details.
\begin{theorem}\label{thm2} 
For any $\epsilon>0$, and integer $k\ge2$, every
$\left(\frac{1}{k-1}+\epsilon\right)$-tough $2K_2$-free graph $G$
admits a $k$-walk.
Moreover, the latter can be found in time polynomial in $|V(G)|$.
\end{theorem}

If we let the toughness value $\frac{1}{k-1}+\epsilon$ increase to 
$\frac{1}{k-2}$ then
one does not need $2K_2$-freeness. Indeed, it is shown in 
\cite{jackson1990k} that
every $\frac{1}{k-2}$-tough graph has a $k$-walk. 
Clearly, if $G$ is
Hamiltonian, then $G$ is 1-tough.  More generally, 
If $G$ has a $k$-walk, then $G$ is $\frac{1}{k}$-tough \cite{jackson1990k}.
However, the converse is not true already for $k=1$.  

This more or less summarises the situation with $t$-tough graphs, $t\leq 1$.
On the $t>1$ side 
a famous conjecture of V.~Chv\'{a}tal \cite{chvatal1973tough} claims
that there exists a constant $\beta$ such that every
$\beta$-tough graph is Hamiltonian.  
Towards this, 
M.~Ellingham and X.~Zha \cite{ellingham2000toughness} proved that
every 4-tough graph has a 2-walk.

Recently, in \cite{broersma2014toughness}, H.~Broersma, V.~Patel and A.~Pyatkin proved that 
every 25-tough 2$K_2$-free graph on at least three vertices is Hamiltonian.
Our Theorem~\ref{thm2} was inspired by this result.  
However, our approach is technically quite different. 

To prove Theorem~\ref{thm2}, we first prove
\begin{theorem}\label{addgen1} 
The  vertex set $V(G)$ of any $2K_2$-free graph $G$
can be partitioned into two parts, $V(F)$ and $V(D)$, where the
induced subgraph $F$ admits a 2-walk having at most one vertex visited twice by
the walk, and the induced subgraph $D$ is a coclique. Moreover, this
partition and the 2-walk in $F$ can be found in time polynomial in $|V(G)|$.
\end{theorem}

The terminology we use is  mostly from \cite{bomu08}; we include the following for the sake of
clarity.  
A {\em clique} means a complete subgraph, a {\em coclique} means a
induced subgraph containing no edge. 
For any
$v\in V$, let $v^\perp:=v^{\perp_G}$ denote the union of $\{v\}$ and the set of
neighbours of $v$ in $G$.  More generally, for $A\subset V$, let
$A^\perp:=A^{\perp_G}$ denote the set of vertices {\em dominated} by $A$, i.e.
$A^\perp=\bigcup\limits_{y\in A}v^\perp$, and $A$ is said to be {\em
dominating} if $A^\perp=V$.  It should also lead to no confusion when we talk
about  dominating induced subgraphs $H$ of $G$, in the sense of $V(H)$ being
dominating, respectively a subset of vertices (e.g. set of vertices of a
subgraph) being dominated by $H$, etc.

\section{Properties of $2K_2$-free graphs and proof of Theorem \ref{addgen1}}
An induced subgraph $H$ of $G$ with $B=V(H)$ (resp. $B\subset V$) is said to be {\em weakly-dominating}
an induced subgraph $F$ (we allow $F=G$ here) 
if for any $v_1v_2\in E(F)$, we have $v_1\in B^\perp$ or $v_2\in B^\perp$.
Further, let $\omega(G)$ denote the maximum size of cliques in $G$.
A {\em blow-up} of $G$ is the result of replacing $v\in V(G)$ 
by a number of vertices $u_1,\dots,u_r$ so that $u_i^\perp-\{u_i\}=v^\perp-\{v\}$
for all $1\leq i\leq r$.


The following properties of $2K_2$-free graphs are very useful for the proof of Theorem \ref{addgen1}.
\begin{lemma}\label{trivial}
$G$ is $2K_2$-free if and only if each edge in $G$ is weakly-dominating. \qed
\end{lemma}

\begin{lemma}\label{lm1}{\cite[Theorem 2]{chung1990maximum}, \cite[Lemma 2]{broersma2014toughness}}
Let $G$ be a connected $2K_2$-free graph with $\omega(G)=2$. Then one of the following holds. 
\begin{enumerate}
\item $G$ is bipartite.
\item $G$ can be obtained from the 5-cycle by repeated blow-ups. \qed
\end{enumerate}
\end{lemma}

First we look at the case of  $\omega(G)=2$ in Theorem \ref{addgen1}.
\begin{theorem}\label{thm4}
Let $H$ be a connected $2K_2$-free graph with $\omega(H)=2$. 
Then $V(H)$ admits a decomposition into two parts $V(F)$ and $V(\Delta)$, so that the induced subgraph $\Delta$ is a coclique, and the induced subgraph $F$ admits
a spanning path, i.e. a path visiting each vertex exactly once, and 
a 2-walk with at most one vertex visited more than once. Both the path and the 2-walk can be found in time polynomial in $|V(G)|$.
\end{theorem}


\begin{proof}
Let $S:=\{S_i\mid 1\leq i\leq  \sigma\}$ be a maximum size matching in $H$.
Denote by $F$ the subgraph spanned by $S$, and
$S_{i}=v_{i}w_{i}$, with $v_i, w_i\in V(H)$, $1\leq i\leq \sigma$. 
As $S$ is maximum, the complement $\Delta$ of $F$ in $H$ is a coclique.

Let $F$ be bipartite, with the parts $\{v_{i}\mid 1\leq i\leq\sigma\}$ and 
$\{w_{i}\mid i=1\leq i\leq\sigma\}$. 
Denote by $F^t$ the induced subgraph of $F$ containing the first $t$ of the edges $S_i$.
We shall prove by induction on $t$ that in $F^t$ there is a spanning path $v'_{1}w'_{1}v'_{2}\cdots v'_{t}w'_{t}$ and each $v'_{p}w'_{p}$ is 
in $S$.

If $t=1$ then $v_{1}w_{1}$ is the path we want. Now, we assume that $F^t$ admits a spanning path, namely $v'_{1}w'_{1}\cdots v'_{t}w'_{t}$, 
with each $v'_{p}w'_{p}\in S$, and consider $F^{t+1}$.

For the case $v'_{t}w_{t+1}\in E(G)$ 
(resp. $v'_{1}w_{t+1}\in E(G)$)
we get a spanning path 
$v'_{1} w'_{1}\cdots w'_{t} v_{t+1} w_{t+1}$
(resp. $v_{t+1} w_{t+2} v'_{1} w'_{1} v'_{2}\cdots w'_{t}$)
in $F^{t+1}$.

If $w'_{t}v_{t+1},v'_{1}w_{t+1}\not\in E(G)$, from the weak dominance
we get $v'_{t}w_{t+1}\in E(G)$ and $w'_{1}v_{t+1}\in E(G)$. Now, let
$q$ be the smallest subscript such that $v'_{q}w_{t+1}\in E(G)$. That
means $v'_{q-1}w_{t+1}\not\in E(G)$. Again, by weak dominance,
we get $w'_{q-1}v_{t+1}\in E(G)$. Thus we get a spanning path in
$F^{t+1}$, namely $$v'_{1}\cdots
w'_{q-1}v_{t+1}w_{t+1}v'_{q}\cdots w'_{t}.$$

By induction, $F$ has a spanning path from $u_1\in V(F)$ to
$u_{2\sigma}\in V(F)$. %, with all the edges in $S$ in the path.

If $\sigma=1$ then $F$ admits a 1-walk,
and we are done.

Now we can assume $\sigma\geq 2$.
If $u_1u_{2\sigma}\in E(F)$ then $F$ is
Hamiltonian, and we are done. 
If we have both $u_1$ and $u_{2\sigma}$ adjacent to non-intersecting edges in $E(H)-E(F)$, 
then we have a contradiction, as $S$ is not maximum.
If we have both $u_1$ and $u_{2\sigma}$ adjacent to  $u\in V(H)-V(F)$, 
then we add $u$ to $F$ and remove it from $\Delta$, obtaining
a Hamiltonian $F$.
Thus we can assume that $u_1$ is not adjacent to  $V(H)-V(F)$, 
while $u_{2\sigma}$ may be adjacent to  $V(H)-V(F)$. 


Now we complete the case $\sigma=2$. 
If there are no edges in $E(H)-E(F)$ adjacent to $u_4$, 
we add $u_1$ and $u_4$ to $\Delta$ and remove them from $F$. 
The resulting $F$ is an edge, and thus has a 1-walk.\\
It remains, without loss in generality,
to consider the case where there exists $u_5\in V(H)-V(F)$ so that
$u_4u_5\in E(H)-E(F)$, while $u_1$ is not
adjacent to $u_4$. As $u_1u_2$ weakly dominates $u_4u_5$, we have that $u_2$ must be 
adjacent to $u_5$. 
We set $F$ to be the subgraph induced by
$u_2,\dots,u_5$, remove $u_5$ from $\Delta$, and add $u_1$ to $\Delta$; then $F$ is Hamiltonian, and we are done.

To finish off the bipartite case it remains to consider $\sigma\geq 3$.
As $u_1u_2$ weakly dominates $u_{2\sigma-1}u_{2\sigma}$, then $u_2u_{2\sigma-1}\in E(F)$, 
otherwise $F$ is
non-bipartite (for adjacent $u_2$ and $u_{2\sigma}$ or $u_1$ and $u_{2\sigma-1}$).
As $u_1$ is not adjacent to   $V(H)-V(F)$, we add it to $\Delta$ and remove from $F$,
obtaining $F$ with a 2-walk where only $u_{2\sigma-1}$ is visited twice. 


\medskip

It remains to consider a non-bipartite $F$. By Lemma \ref{lm1},
$F$ can be obtained from a 5-circle by ``vertex multiplication''. We
denote $V(F)=V_0\cup \dots \cup V_4$, and
$E(F)=E_0\cup \dots \cup E_4$, where each $V_i$ is an
independent set, and  for any $a\in V_i$ and $b\in V_{i+1}$ we have $ab\in E_i$. 
Here and below we always take the indices of $V_i$, $E_i$, etc. modulo $5$.
Let $S=\{v_{j,1}v_{j,2}\mid 1\leq j\leq \sigma\}$ be a maximum matching in $F$, and $E'_i:=E_i\cap S$.
Obviously,
$|V_i|=|E'_{i-1}|+|E'_i|$. Note $|V_i|>0$ for $0\leq i\leq 4$, otherwise
$F$ is bipartite. Thus $|E'_i|+|E'_{i-1}|>0$ for any $0\leq i\leq 4$. 
Without loss of generality, we can assume $|E'_0||E'_2||E'_4|>0$.
Note that $E'_1$ and $E'_3$ can be empty.

$V_i$ is partitioned into
$V_i^-:= V_i\cap V(E'_{i-1})$ and 
$V_i^+:= V_i\cap V(E'_{i})$. For each $\Sigma:=E'_i\neq\emptyset$ there is
a path $\Pi_i$ connecting its vertices in $V_i$ and in $V_{i+1}$, 
as follows: $u_1 w_1 u_2\dots u_{|\Sigma|}w_{|\Sigma|}$, with 
$u_k\in V_i^+$ and $w_k\in V_{i+1}^-$.%\footnote{If we wish we can choose $u_k w_k$ to be
%a $Q_{m+j}$.} 
Note that there is an edge joining the last vertex of $\Pi_i$ and
the first vertex of $\Pi_{i+2}$. Thus it makes sense to talk about the path
$\Pi_i\Pi_{i+2}$, which joins $V_i^+$, $V_{i+1}^-$, $V_{i+2}^+$ and
$V_{i+3}^-$.

In the case of both $E'_1$ and $E'_3$ nonempty, we observe that 
$\Pi_0\Pi_2\Pi_4\Pi_1\Pi_3$ is a Hamiltonian cycle, as the end of
$\Pi_3$ is adjacent to the beginning of $\Pi_0$. 
If $E'_3=\emptyset$  
we instead use the path 
$\Pi_4\Pi_1\overline{\Pi}_2\overline{\Pi}_0$, where $\overline{\Pi}_\alpha$
denotes $\Pi_\alpha$ taken backwards, to obtain a Hamiltonian
cycle, using the fact that the beginnings of $\Pi_0$ and $\Pi_4$ are
adjacent.

If $E'_1\cup E'_3=\emptyset$ 
we can take the Hamiltonian path $\Pi_0\Pi_2\Pi_4$, 
which begins and ends at $V_0$. Denote the first
edge of this path by $ab$ and the last vertex of it 
by $c$. As $c\in V_0$ and $b\in V_1$, they are adjacent.
Thus we have a 2-walk in $F$, where only $b$ is visited
twice.

Finally, we note that the only nontrivial algorithm we need is the one to find a maximum matching, and this is 
well-known to be doable in polynomial time using Edmonds' algorithm, cf. e.g. \cite[Sect. 16.5]{bomu08}.
\end{proof}

To complete the proof of Theorem \ref{addgen1} in the general case
we will use the following decomposition of $V(G)$.
\begin{theorem}\label{thm1}
Let $G$ be a connected $2K_2$-free graph. Then $V(G)$ 
can be partitioned into $V(Q_0)$, $V(Q_{\infty})$ and $V(D)$ (some of these parts possibly empty), so that for the corresponding induced subgraphs
$Q_0$, $Q_\infty$, and $D$ the following holds: 
\begin{enumerate}[(\ref{thm1}.1)]
\item $Q_0$ is dominating in $G$ and Hamiltonian, \label{i51}
\item $D$ is a coclique,\label{i52}
\item $Q_{\infty}$ admits a spanning 2-walk with at most one vertex visited more than once, \label{i53}
\item $Q_{\infty}$ admits a spanning path (a.k.a. Hamiltonian path).\label{i54}
\end{enumerate}
\end{theorem}

To prove the latter, by giving an (efficient) algorithm, we use the following theorem, which is an
efficient version of \cite[Theorem 3]{chung1990maximum}.
\begin{theorem}\label{lm2}
If $G$ is $2K_2$-free with $\omega(G)\ge3$, then a maximal (but not necessarily maximum) dominating clique $K$ in $G$ of size at least 3
can be computed in polynomial time.
\end{theorem}

\begin{proof}
First, find a clique of size 3,  and enlarge it to a maximal clique $K=\{x_1,x_2,\ldots,x_p\}$ in $G$, by 
the greedy procedure.
Let $Z:=Z(K):=V(G)-K^{\perp}$. 
If $Z=\emptyset$, then $K$ is dominating, and we are done. So we may assume $Z\neq\emptyset$.
Now we describe an iteration that either produces a clique of size bigger than $p$, or
a clique $K'$ of size $p$ with $|Z(K')|<|Z(K)|$.

As $p\ge2$, we have that $Z$ is a coclique. 
For each $1\leq i\leq p$, let 
$Y_i=K^\perp - \cup_{x\in K-\{x_i\}} x^\perp$.
Note that $Y_i\cap Y_j=\emptyset$ for $i\neq j$. 
Since $p\ge3$, each $Y_i$ is a coclique, due to $2K_2$-freeness.
 
Choose an arbitrary $z_0\in Z$ and let $y_0\in K^{\perp}-K$ be a neighbour of
$z_0$. Since $G$ is $2K_2$-free and $p$ is maximal, $y_0\in Y_i$ for
some $i$.  Therefore
$K'=(K-\{x_i\})\cup\{y_0\}$ is a clique of size $p$. 
If $K'$ is not maximal we obtain a clique stricly larger than $K$.

Any vertex dominated by $K$ is dominated by $K'$, except these in
$Y':= Y_i-y_0^\perp$. If $Y'=\emptyset$ then $|Z(K')|<|Z(K)|$, 
as $K'$ dominates all the vertices that are dominated by $K$, 
as well as $z_0$.

Otherwise, let $y_1\in Y'$. Then the edges $z_0y_0$ and $x_iy_1$ force $z_0y_1\in E(G)$, by $2K_2$-freeness. 
As $p\geq 3$, there exist $x_j, x_k\in K\cap K'$, and 
$z_0y_1$ and $x_jx_k$ form $2K_2$,  contradiction, completing the description of the iteration.

We need to repeat the iteration no more than $n$ times to get a dominating clique, and each iteration 
obviously runs in polynomial time.
\end{proof}


\begin{lemma}\label{lm3}{\cite[Observation 1]{broersma2014toughness}}
A graph $G=(V,E)$ is $2K_2$-free if and only if for every $A\subset V$, at most one component of the graph $G-A$ contains edges. \qed
\end{lemma}

\begin{lemma}\label{lm5}
In a $2K_2$-free graph $G=:G_0$, there exists a sequence of cliques
$Q_i\subseteq G_{i-1}$ of size at least $2$, cocliques $D_i\subseteq G_{i-1}$
with $V(Q_i)\cap V(D_i)=\emptyset$,  
and connected subgraphs $G_{i}:=(G_{i-1}-Q_{i})-D_{i}$, with
no edges between $D_i$ and $G_i$,  for
$i=1,\ldots,m+s$, where
$|Q_i|\ge3$ for $1\le i\le m$ and $|Q_i|=2$ for $i=m+1,\ldots,m+s$.
Moreover, $Q_i$ is a maximal
(respectively, weakly-) dominating
clique in $G_{i-1}$ for $1\leq i\leq m$ (respectively, for $i>m$).
\end{lemma}
\begin{proof}
We construct the claimed sequences of $Q_i$, $D_i$, 
and $G_i$ directly by induction, as follows. 
We may assume that $G_i$ is $2K_2$-free.
If $\omega(G_i)\ge3$ then by Theorem \ref{lm2}, it contains a 
maximal dominating clique $Q_{i+1}$.
As any induced subgraph of a $2K_2$-free
graph is also $2K_2$-free, $G_i-Q_{i+1}$ is also a $2K_2$-free graph.
By Lemma \ref{lm3}, $G_i-Q_{i+1}$ is made of two parts, one is a coclique
(possibly empty) $D_{i+1}$, and the other is a connected
component (either empty or with a least two vertices) $G_{i+1}$, which is also
$2K_2$-free (again, as an induced subgraph of a $2K_2$-free graph).
Thus from $G_i$ we have constructed $Q_{i+1}$, $D_{i+1}$, and $G_{i+1}$
with the  claimed properties.

We repeat this step until $\omega(G_i)=2$ (or $G_i=\emptyset$).
Thus $i=m$ at the end of this loop.
Applying Theorem~\ref{thm4} to $G_m$, we obtain
a connected subgraph $F$.% and a coclique $\Delta$.

Select a maximum matching $Q_{m+1},\dots, Q_{m+s}$ in $F$
and use it to complete the sequences of $D_i$ and $G_i$. 
\end{proof}


\begin{definition}\label{lm6}
The vertex set of a $2K_2$-free graph $G$ can be divided into three disjoint, 
possibly empty, parts:
\begin{enumerate}
\item $V(Q_0)=\cup_{i=1,\ldots,m}V(Q_i).$\label{Vq0}
\item $V(Q_{\infty})=\cup_{i=m+1,\ldots,m+s}V(Q_i).$\label{Vqinfty}
\item $V(D)=\cup_{i=1,\ldots,m+s}V(D_i)$\label{Vd}
\end{enumerate}
Here, $Q_0$ is the induced subgraph of $G$ on $V(Q_0)$, $Q_{\infty}$ is the induced subgraph of $G$ on $V(Q_{\infty})$ and $D$ is the induced subgraph of $G$ on $V(D)$. 
Denote $V(Q)=V(Q_0)\cup V(Q_{\infty})$, where $Q$ is the induced subgraph of $G$ on $V(Q)$. 
\end{definition}
Note that obviously $V(G)=V(Q)\cup V(D)$.

\subsection{Proof of Theorem \ref{thm1}}
Note that Theorem \ref{thm4} applied to $Q_\infty$ implies (\ref{thm1}.\ref{i53}) and  (\ref{thm1}.\ref{i54}).
By Lemma~\ref{lm5}, there are no edges between $D_i$ and $G_i$. By construction, 
$D_j$, for $j>i$, lies in $G_i$. Thus $D$ is a coclique.
It remains to prove that $Q_0$ is Hamiltonian.

Let $\{Q_i\}$ and $m$ be as defined in Lemma \ref{lm5}.
\begin{lemma}\label{lem:twoQ}
Let $Q_i$ and $Q_j$ so that $i<j$ and $|Q_i|\geq 3$.
Then for any $a_j b_j\in E(Q_j)$  there is $a_i b_i\in E(Q_i)$ satisfying
$a_i a_j, b_i b_j\in E$.
\end{lemma}
\begin{proof}
Let $ab\in E(Q_j)$. As $Q_i$ dominates $Q_j$, in $Q_i$ there is a vertex adjacent to 
$a$ and a vertex adjacent to $b$.
We claim that on the other hand $a$ and $b$ are adjacent to at least $|Q_i|-1$ vertices in $Q_i$. 
Indeed, if there were $c\neq d\in V(Q_i)$ adjacent neither to $a$ nor to $b$, 
then $ab$ and $cd$ would form a $2K_2$.
As $|Q_i|-1\ge 2$, we can choose $cd\in E(Q_i)$ such that $ac, bd\in E$.
\end{proof}
 
The rest of Theorem \ref{thm1} is immediate from the first part of the following.
\begin{lemma}\label{lem:Q0H}
The subgraph $Q_0$ is Hamiltonian. 
For any $v\in V(Q_1)$ and $w\in V(Q_m)$ there exists a Hamiltonian
path in $Q_0$ with endpoints $v$ and $w$.
\end{lemma} 
\begin{proof}
If $m=0$, then $Q_0=\emptyset$ and there is nothing to prove.
If $m=1$, i.e. $Q_0=Q_1$, then obviously $Q_0$ is Hamiltonian, since it is just a clique, and
has a Hamiltonian path joining any two vertices.\\
If $m=2$, i.e. $V(Q_0)=V(Q_1)\cup V(Q_2)$, then by Lemma~\ref{lem:twoQ} 
for a $ab\in E(Q_2)$
we can find $cd\in E(Q_1)$ so that $ac,~bd\in E(G)$. 
Note that we can assume $w\neq a$, $w\neq b$, as $|V(Q_2)|\geq 3$.
There exists a Hamiltonian path in  $Q_1$ (respectively in $Q_2$) 
with endpoints $c$, $d$ (respectively $a$, $b$).
These two paths and $ac$ and $bd$ form a Hamiltonian cycle.
Without loss in generality, $c\neq v$. Hence we can construct a
Hamiltonian path in $Q_0$ by concatenating a $Q_1$-Hamiltonian
path $cv$, the  edge $av$, and a $Q_2$-Hamiltonian path $wa$.
 
It remains to deal with $m\geq 3$.
Applying Lemma~\ref{lem:twoQ} to $Q_{i-1}$ and $Q_i$ for each $2\leq i\leq m$, 
we construct $a_{i-1}a_i, b_{i-1}b_i\in E$ joining $Q_{i-1}$ and $Q_i$.
Select a half of them, namely $a_{m}a_{m-1}$, $b_{m-1}b_{m-2}$, $a_{m-2}a_{m-3}$, \dots,
$a_2a_1$ in the case $m$ even (respectively, $b_2b_1$ for $m$ odd).

Next, construct an edge between $Q_m$ and $Q_1$, avoiding vertices of the edges just
selected, i.e. $a_m\in V(Q_m)$ and $x\in V(Q_1)$. 
This can be done by applying Lemma~\ref{lem:twoQ} to $Q_m$, $Q_1$, 
and $a,b\in V(Q_m)-\{a_m\}$ to obtain $a'b'\in E(Q_1)$ so that $aa',bb'\in E$, 
noticing that either $aa'$ or $bb'$ will be as needed. Say, for $m$ even 
it is $aa'$
that satisfies $a'\neq a_1$, (respectively, $a'\neq b_1$ for $m$ odd)  and we 
abuse notation by denoting $b_m:=a$ and $b_1:=a'$ (respectively for $m$ odd 
we denote $a_1:=a'$).

Finally, we construct Hamiltonian paths in each $Q_i$ joining the pairs of vertices
$a_i,b_i\in V(Q_i)$ just constructed, obtaining the required Hamiltonian cycle.

To prove the second part of the lemma for $m\geq 3$, observe that we can choose
$w\neq a_m$, $w\neq b_m$. Without loss in generality we can assume that $v\neq a_1=:a$ 
for $m$ even (resp. $v\neq b_1=:a$ for $m$ odd), 
and proceed to construct the Hamiltonian path
which first goes from $w$ to $a_m$ in $Q_m$, then goes the same way as the Hamiltonian
path we just constructed to $a$, and then joins $a$ and $v$ in $Q_1$ by a Hamiltonian path 
in the latter.
\end{proof}

\subsection{Proof of Theorem \ref{addgen1}}
It remains to consider the case $\omega(G)\ge3$.
Recall the sequence of cliques $\{Q_i\}$, $m$ and $s$ as defined in Lemma \ref{lm5}.


If $s=0$ we have $Q=Q_0$, since $Q_{\infty}=\emptyset$. Then by Theorem \ref{thm1}, $Q$ is Hamiltonian.
Now let $s\geq 1$. 
Theorem \ref{thm1} (\ref{thm1}.\ref{i54}) provides 
a spanning path in $Q_{\infty}$, say $\Pi:=u_1\cdots u_{2s}$. 

If $m\ge2$ then, 
as each vertex in $Q_{\infty}$ is dominated by $Q_i$, for $1\le i\le m$, we
can find $v\in Q_1$ and $w\in Q_m$ such that
$u_1v,~u_{2s}w\in E(G)$. By Lemma~\ref{lem:Q0H}, there is a Hamiltonian path  $v\cdots w$ in $Q_0$.
Joining it to $\Pi$, we obtain a Hamiltonian cycle in $Q$.

We have just proved the following.
\begin{lemma}\label{lem:HamQ}
If $s=0$ or $m\geq 2$ then $Q$ is Hamiltonian. \qed
\end{lemma}
It remains to consider {$m=1$}.
If $u_1$ and $u_{2s}$ have two different neighbours $v_1$ and $v_2$ in $Q_1$
then we can pick up a spanning path in $Q_1$ with $v_1$ and $v_2$ as two endpoints. 
Connecting the latter to $\Pi$, we get a Hamiltonian cycle in $Q$. \\
If $u_1$ and $u_{2s}$ are adjacent to only one vertex, say $v_{\infty}$ in $Q_1$, 
then a Hamiltonian cycle in $Q_1$
connected to $\Pi$ by $v_{\infty}$ gives a 2-walk in $Q$, with $v_{\infty}$ visited twice and all other vertices visited once. 

Finally, we note all the constructions we needed can be done in time polynomial in $|V(G)|$.
\qed

\section{Proof of Theorem \ref{thm2}}
The $(\frac{1}{k-1}+\epsilon)$-toughness condition with $\epsilon>0$ 
is only used for the dominating cliques sequence $\{Q_i\}$ 
and parameters $m$ and $s$, as defined in Lemma \ref{lm5},
in the case $m=1$ and $s\geq 1$. Namely, we prove the following
slightly stronger
\begin{theorem}\label{opt}
Let $G$ be a connected $2K_2$-free graph.
If $G$ is $\left(\frac{1}{k-1}+\epsilon\right)$-tough, for an
$\epsilon>0$, then $G$ admits a $k$-walk. 
Moreover, a $\frac{1}{k-1}$-tough $G$ admits a $k$-walk if either
$m\ge2$, or if $G$ is {\em split}, i.e. $m=1$, $s=0$.  
\end{theorem}




The following lemma is the key technique in the proof of Theorem \ref{thm2}.

\begin{lemma}\label{addtec}
Let $V(G)$ have a partition $V(G)=V(Q)\cup V(D)$, 
so that the induced subgraph $D$ is a coclique.
\begin{enumerate} 
\item If the induced subgraph $Q$ is Hamiltonian 
and $G$ is $\frac{1}{k-1}$-tough, 
then $G$ admits a $k$-walk.\label{itone}
\item If the induced subgraphs $Q$ admits a 2-walk, 
where only one vertex $v_{\infty}$ is visited twice in the walk 
and all other vertices are visited once, and $G$ is $(\frac{1}{k-1}+\epsilon)$-tough, then $G$ admits a $k$-walk.\label{ittwo}
\end{enumerate}
\end{lemma}

\begin{proof}
Let $D_0\subset D$.
For part (\ref{itone}), 
by $\frac{1}{k-1}$-toughness, 
$D_0$ has at least $\left\lceil\frac{|D_0|}{k-1}\right\rceil$ neighbours in $Q$. By Hall's Theorem, 
there is $E'\subset E(G)$ such that each $e\in E'$ has one vertex in $D$ and the other in $Q$,
and each vertex in $D$ is incident to exactly one edge in $E'$, while each vertex in $Q$ is incident to at most $k-1$ edges in $E'$.
Then these ({doubled}) edges in $E'$ and the edges in the Hamiltonian cycle $H$ of $Q$ form a $k$-walk in $G$.

For part (\ref{ittwo}), by $\left(\frac{1}{k-1}+\epsilon\right)$-toughness, 
$D_0$ has at least $\left\lceil(\frac{1}{k-1}+\epsilon)|D_0|\right\rceil$ neighbours in $Q$. 
Split each vertex in $Q$ into $k-1$ copies and denote the set of all the split vertices by $Q^{k-1}$. 
Then $D_0$ has at least $(k-1)\lceil(\frac{1}{k-1}+\epsilon)|D_0|\rceil\ge|D_0|+1$ neighbours in $Q^{k-1}$. 
Pick one copy of $v_{\infty}$, namely $v^*\in Q^{k-1}$. Then $D_0$ has at least $|D_0|$ neighbours 
in $Q^{k-1}-\{v^*\}$. Thus by Hall's Theorem there is a matching from $D$ to $Q^{k-1}-\{v^*\}$. 
Now collapse these copies back to $Q$. Each vertex in $D$ is incident to exactly one edge from the resulting matching $E'$. 
Each vertex in $Q-\{v_{\infty}\}$ takes at most $k-1$ edges from $E'$,
while $v_{\infty}$ takes at most $k-2$ edges from $E'$. 
Then $2*E'$ and the edges from the 2-walk in $Q$ form a $k$-walk in $G$.
\end{proof}


Combining Lemmas \ref{addtec} and \ref{lem:HamQ}, and Theorem \ref{addgen1},
we obtain Theorem \ref{opt}.

Note that the only nontrivial algorithmic part of the proof of Theorem
\ref{opt}  not already covered is the construction of a maximum matching in
certain blowup (of size polynomially bounded by $|V(G)|$) of a bipartite
subgraph of $G$, and this is well-known to be doable in time polynomial in
$|V(G)|$, cf. e.g. \cite[Sect. 16]{bomu08}.  This completes the proof of
Theorem \ref{thm2}.

\medskip

Finally, let us state two conjectures.
\begin{conjecture}
Every $3/2$-tough $2K_2$-free graph with at least 3 vertices has a 2-trail, i.e. a 2-walk with each edge appearing in the walk at most once.
\end{conjecture}


\begin{conjecture}
Every 2-tough $2K_2$-free graph with at least 3 vertices is Hamiltonian.
\end{conjecture}

\subsection*{Acknowledgements.}
The authors thank Nick Gravin
for helpful comments on a draft of this text.
Research supported by Singapore MOE Tier 2 Grant MOE2011-T2-1-090 (ARC 19/11). 

\bibliography{reftough}
\bibliographystyle{abbrv}
%\bibliographystyle{plain}
\end{document}
