\documentclass{amsart}
\usepackage{amsthm}
\usepackage{enumerate}

\theoremstyle{definition}
\newtheorem{theorem}{Theorem}
\newtheorem{problem}{Problem}
\newtheorem{remarks}{Remarks}
\newtheorem{conjecture}{Conjecture}
\newtheorem{lemma}{Lemma}
\newtheorem{step}{Step}
\newtheorem{case}{Case}
\newtheorem{definition}{Definition}
\newcommand{\ncc}{c}

\begin{document}
\author{Gao Mou}
\address{School of Physical and Mathematical Sciences, Nanyang Technological University, Singapore} 
\email{gaom0002@e.ntu.edu.sg} 
\author{Dmitrii V. Pasechnik}
\address{Department of Computer Science, The University of Oxford, UK}
\email{dimpase@cs.ox.ac.uk}

\title{On $k$-walks in $2K_2$-free graphs}
\begin{abstract}
We show that for any $\epsilon>0$ every $(\frac{1}{k-1}+\epsilon)$-tough
$2K_2$-free graph admits a $k$-walk, making a step towards proving a
long-standing conjecture from \cite{jackson1990k}. The proof uses a
decomposition of $2K_2$-free graphs which might be of independent interest.
\end{abstract}

\maketitle

\section{Introduction}

Let  $\ncc(G)$ denote the number of connected components of a graph $G$ with vertex set $V:=V(G)$ and edge set $E:=E(G)$. For $S\subset V$, let $G-S$ denote the subgraph of $G$ induced on $V-S$.
\begin{definition}
$G$ is called $\beta$-{\em tough},
for a real $\beta>0$, if $\ncc(G-S)>1$ implies $|S|\ge \beta\cdot \ncc(G-S)$ 
for each $S\subset V$.
\end{definition}
That is, a $\beta$-tough 
$G$ cannot be split into $p$ (with $p>1$) components by removing less than
$p\beta$ vertices.  
This concept, a measure of graph connectivity and ``resilience'' under vertex subsets removal,
was introduced in 1973 by V. Chv\'{a}tal 
while studying   Hamiltonicity of graphs. For a survey of results on graph toughness till 2006
cf. \cite{MR2221006}. Our graph-theoretic notation is mostly from \cite{bomu08}.

In 1990 B.~Jackson and N.~Wormald conjectured \cite{jackson1990k} that every
$\frac{1}{k-1}$-tough graph $G$ admits a $k$-walk, i.e. a sequence of edges
visiting each vertex at least once and at most $k$-times, for any integer $k\ge2$. 

In this paper, we prove this conjecture is ``almost'' true, under the
assumption that $G$ is  $2K_2$-free, that is $G$ does not contain an induced
copy of $2K_2$, i.e. the disjoint union of two edges.  Here by ``almost'' we
mean that we need to replace $\frac{1}{k-1}$-tough by
$(\frac{1}{k-1}+\epsilon)$-tough, with $\epsilon>0$ arbitrarily small.  In fact
$\epsilon$ can be set to 0 for most $2K_2$-free graphs, and the outcome of
Theorem~\ref{thm2} will remain true, cf. Theorem~\ref{opt} below for details.
\begin{theorem}\label{thm2} 
For any $\epsilon>0$, and integer $k\ge2$, every
$(\frac{1}{k-1}+\epsilon)$-tough $2K_2$-free graph admits a $k$-walk.
\end{theorem}

If we let the toughness value $\frac{1}{k-1}+\epsilon$ to increase to 
$\frac{1}{k-2}$ then
one does not need $2K_2$-freeness. Indeed, the following holds.
\begin{theorem}\cite{jackson1990k}\label{cthm1}
Every $\frac{1}{k-2}$-tough graph has a $k$-walk. 
In particular every 1-tough graph has a 3-walk. \qed
\end{theorem}
Clearly, if $G$ is
Hamiltonian, then $G$ is 1-tough.  More generally:
\begin{theorem}\cite{jackson1990k}\label{add}
If $G$ has a $k$-walk, then $G$ is $\frac{1}{k}$-tough. \qed
\end{theorem} 
However, the converse is not true already for $k=1$.  

This more or less summarises the situation with $t$-tough graphs, $t\leq 1$.
On the $t>1$ side 
a famous conjecture of V.~Chv\'{a}tal \cite{chvatal1973tough} claims
that there exists a constant $\beta$ such that every
$\beta$-tough graph is Hamiltonian.  

The concept of a $k$-walk is a generalization of the concept of a Hamiltonian cycle.
Ellingham and Zha \cite{ellingham2000toughness} proved that
every 4-tough graph has a 2-walk.

Recently, in \cite{broersma2014toughness}, H.~Broersma, V.~Patel and A.~Pyatkin proved that 
every 25-tough 2$K_2$-free graph on at least three vertices is Hamiltonian.
Our Theorem~\ref{thm2} was inspired by this result.  
However, our
approach is technically quite different, and, unlike \cite{broersma2014toughness}, we do not seem to be
able to make any good algorithmic conclusions about the problem at hand.
Indeed, the construction relies on finding maximum cliques in
$2K_2$-free graphs, and this is known to be NP-hard. 
To prove Theorem~\ref{thm2}, we first prove
\begin{theorem}\label{thm1}
For any $2K_2$-free graph $G$ the set $V(G)$ 
can be partitioned into three disjoint parts $V(Q_0)$, $V(Q_{\infty})$ and $V(D)$ (some of these parts possibly empty), so that for the corresponding induced subgraphs
$Q_0$, $Q_\infty$, and $D$ the following holds: $Q_0$ is dominating in $G$ and Hamiltonian, 
$Q_{\infty}$ admits a spanning path (also called Hamiltonian path), 
and $D$ is a coclique.
\end{theorem}

The terminology just used is standard; we include the following for the sake of
clarity.  A {\em clique} means a complete subgraph, a {\em coclique} means a
induced subgraph containing no edge. For $A=\{v_1,\ldots,v_m\}\subseteq V$, a
subgraph $P$ in $G$ with $V(P)=A$ and
$E(P)=\{v_1v_2,v_2v_3,\ldots,v_{m-1}v_m\}$ is called a {\em path}.  For any
$v\in V$, let $v^\perp:=v^{\perp_G}$ denote the union of $\{v\}$ and the set of
neighbors of $v$ in $G$.  More generally, for $A\subset V$, let
$A^\perp:=A^{\perp_G}$ denote the set of vertices {\em dominated} by $A$, i.e.
$A^\perp=\bigcup\limits_{y\in A}v^\perp$, and $A$ is said to be {\em
dominating} if $A^\perp=V$.  It should also lead to no confusion when we talk
about  dominating induced subgraphs $H$ of $G$, in the sense of $V(H)$ being
dominating, respectively a subset of vertices (e.g. set of vertices of a
subgraph) being dominated by $H$, etc.

\section{Properties of $2K_2$-free Graphs and Proof of Theorem \ref{thm1}}
An induced subgraph $H$ of $G$ with $B=V(H)$ (resp. $B\subset V$) is said to be {\em weakly-dominating}
an induced subgraph $F$ (we allow $F=G$ here) 
if for any $v_1v_2\in E(F)$, we have $v_1\in B^\perp$ or $v_2\in B^\perp$.
Further, let $\omega(G)$ denote the maximum size of cliques in $G$. 

\subsection{Properties of $2K_2$-free Graphs}
The following properties of $2K_2$-free graphs are very useful in the proof of Theorem \ref{thm1} and Theorem \ref{thm2}.

\begin{lemma}\label{trivial}
$G$ is $2K_2$-free if and only if each edge in $G$ is weakly-dominating. \qed
\end{lemma}

\begin{lemma}\label{lm1}{\cite[Theorem 2]{chung1990maximum}, \cite[Lemma 2]{broersma2014toughness}}
Assume that $G$ is $2K_2$-free, $\omega(G)=2$, then $G$ is in one of the following cases.
\begin{enumerate}
\item $G$ is a bipartite.
\item $G$ can be obtained from a five-circle by vertex multiplication. \qed
\end{enumerate}
\end{lemma}

\begin{lemma}\label{lm2}{\cite[Theorem 3]{chung1990maximum}}
If $G$ is $2K_2$-free with $\omega(G)\ge3$, then $G$ has a dominating clique of size $\omega(G)$. \qed
\end{lemma}

\begin{lemma}\label{lm3}{\cite[Observation 1]{broersma2014toughness}}
A graph $G=(V,E)$ is $2K_2$-free if and only if for every $A\subset V$, at most one component of the graph $G-A$ contains edges. \qed
\end{lemma}

\begin{lemma}\label{lm5}
In a $2K_2$-free graph $G=:G_0$, there exists a sequence of cliques
$Q_i\subseteq G_{i-1}$ of size at least $2$, cocliques $D_i\subseteq G_{i-1}$
with $V(Q_i)\cap V(D_i)=\emptyset$,  
and connected subgraphs $G_{i}:=(G_{i-1}-Q_{i})-D_{i}$, with
no edges between $D_i$ and $G_i$,  for
$i=1,\ldots,m+s$, where
$|Q_i|\ge3$ for $1\le i\le m$ and $|Q_i|=2$ for $i=m+1,\ldots,m+s$.
Moreover, $|Q_i|\ge|Q_{i+1}|$ for any $i$, $Q_i$ is a maximum
(respectively, weakly-) dominating
clique in $G_{i-1}$ for $1\leq i\leq m$ (respectively, for $i>m$).
\end{lemma}
\begin{proof}
We construct the claimed sequences of $Q_i$, $D_i$, 
and $G_i$ directly by induction, as follows. 
We may assume that $G_i$ is $2K_2$-free.
If $\omega(G_i)\ge3$ then by Lemma \ref{lm2}, it contains a 
maximum dominating clique $Q_{i+1}$.
Otherwise it is triangle-free, and any edge $Q_{i+1}$ is weakly-dominating
in $G_i$.
As any induced subgraph of a $2K_2$-free
graph is also $2K_2$-free, $G_i-Q_{i+1}$ is also a $2K_2$-free graph.
By Lemma \ref{lm3}, $G_i-Q_{i+1}$ is made of two parts, one is a coclique
(possibly empty) $D_{i+1}$, and the other is a connected
component (either empty or with a least two vertices) $G_{i+1}$, which is also
$2K_2$-free (again, as an induced subgraph of a $2K_2$-free graph).
Thus from $G_i$ we have constructed $Q_{i+1}$, $D_{i+1}$, and $G_{i+1}$
with the  claimed properties.
\end{proof}


\begin{definition}\label{lm6}
The vertex set of a $2K_2$-free graph $G$ can be divided into three disjoint parts:
\begin{enumerate}
\item $V(Q_0)=\cup_{i=1,\ldots,m}V(Q_i).$\label{Vq0}
\item $V(Q_{\infty})=\cup_{i=m+1,\ldots,m+s}V(Q_i).$\label{Vqinfty}
\item $V(D)=\cup_{i=1,\ldots,m+s}V(D_i)$\label{Vd}
\end{enumerate}
Here, $Q_0$ is the induced subgraph of $G$ on $V(Q_0)$, $Q_{\infty}$ is the induced subgraph of $G$ on $V(Q_{\infty})$ and $D$ is the induced subgraph of $G$ on $V(D)$. 
Denote $V(Q)=V(Q_0)\cup V(Q_{\infty})$, where $Q$ is the induced subgraph of $G$ on $V(Q)$. 
\end{definition}
Note that obviously $V(G)=V(Q)\cup V(D)$, and we allow any of these sets 
in (\ref{Vq0})-(\ref{Vd}) to be empty.

\begin{lemma}\label{lmDcoclique}
$D$ is a coclique in $G$.
\end{lemma}
\begin{proof}
By Lemma~\ref{lm5}, there are no edges betwwen $D_i$ and $G_i$. By constructuon, 
$D_j$, for $j>i$, lies in $G_i$.
\end{proof}

\subsection{Spanning path in $Q_{\infty}$.}
Now, let us look at the behavior of all the cliques of size 2.
\begin{theorem}\label{thm4}
For a $2K_2$-free graph $G$, $Q_{\infty}$, cf. Definition~\ref{lm6}, is the induced subgraph on all the vertices of all the 2-cliques in a maximum (weakly) dominating sequence. 
Then $Q_{\infty}$ admits a spanning path, i.e. a path visiting each vertex in $Q_{\infty}$ exactly once.
\end{theorem}


\begin{proof}
Denote $Q_{m+i}=v_{i,1}v_{i,2}$ for $i=1,\ldots,s$. By definition, $\omega(Q_{\infty})=2$.
\begin{enumerate}
\item If $Q_{\infty}$ is a bipartite, denote the two independent part as $B_1=\{v_{i,1};i=1,\ldots,s\}$ and $B_2=\{v_{i,2};i=1,\ldots,s\}$. 


Denote $Q^t_{\infty}$ as an induced subgraph of $Q_{\infty}$ containing $t$ of the edges $\{Q_{m+i}:i=1,\ldots,s\}$, i.e. let $V(Q^t_{\infty})=\{v_{i,j}:i=1,\ldots,t;~j=1,2\}$
Now let us make an induction on the size $t$. We shall prove in each of the $Q^t_{\infty}$, $(t=1,\ldots,s)$, there is a spanning path $v'_{1,1}v'_{1,2}v'_{2,1}\cdots v'_{t,1}v'_{t,2}$ and each edge $v'_{p,1}v'_{p,2}$ is some edge $Q_{m+q}=v_{q,1}v_{q,2}$.


If $t=1$, then $v_{1,1}v_{1,2}$ is the path we want. Now, we assume that $Q^t_{\infty}$, whose vertex set $V(Q^t_{\infty})=\cup_{i=1,\ldots,t}V(Q_{m+i})$, admits a spanning path, namely $v'_{1,1}v'_{1,2}\cdots v'_{t,1}v'_{t,2}$, with each $v'_{p,1}v'_{p,2}=v_{q,1}v_{q,2}$ for some $q$. Then let us look at a larger induced subgraph $Q^{t+1}_{\infty}$, with $V(Q^{t+1}_{\infty})=V(Q^t_{\infty})\cup\{v_{t+1,1},v_{t+1,2}\}$.

If $v'_{t,2}v_{t+1,1}\in E(G)$, we get the spanning path in $Q^{t+1}_{\infty}$, 
namely\\ $v'_{1,1} v'_{1,2}\cdots v'_{t,2} v_{t+1,1} v_{t+1,2}$.

If $v'_{1,1}v_{t+1,2}\in E(G)$, we get the spanning path in $Q^{t+1}_{\infty}$, 
namely\\ $v_{t+1,1} v_{t+2,2} v'_{1,1} v'_{1,2} v'_{2,1}\cdots v'_{t,2}$.

If $v'_{t,2}v_{t+1,1},v'_{1,1}v_{t+1,2}\not\in E(G)$, from the weakly-dominating property we get $v'_{t,1}v_{t+1,2}\in E(G)$ and $v'_{1,2}v_{t+1,1}\in E(G)$. Now, we denote $q$ as the smallest subscript for $v'_{q,1}v_{t+1,2}\in E(G)$. That means $v'_{q-1,1}v_{t+1,2}\not\in E(G)$. Again, by the weakly-dominating property, we get $v'_{q-1,2}v_{t+1,1}\in E(G)$. Then, we get a spanning path in $Q^{t+1}_{\infty}$, namely $$v'_{1,1}\cdots v'_{q-1,2}v_{t+1,1}v_{t+1,2}v'_{q,1}\cdots v'_{t,2}.$$

Thus, by mathematical induction, we know that $Q_{\infty}$ has a spanning path, with all the edges $Q_{m+i},i=1,\ldots,s$ are in the path.

\item If $Q_{\infty}$ is not a bipartite, by Lemma \ref{lm1}, $Q_{\infty}$ can be obtained from a five-circle by vertex multiplication. We denote $V(Q_{\infty})=V_1\cup V_2\cup V_3\cup V_4\cup V_5$, and $E(Q_{\infty})=E_1\cup E_2\cup E_3\cup E_4\cup E_5$, where each $V_i$ is an independent set, and  for any two vertices $v_1\in V_i$ and $v_2\in V_{i+1}$ (we consider $V_6=V_1$), we have an edge $e\in E_i$, such that $e=v_1v_2$. We denote $E'=\{v_{i,1}v_{i,2};i=1,\ldots,s\}$, and $E'_i=E_i\cap E'$. Obviously, $|V_i|=|E'_{i-1}|+|E'_i|$. Note $|V_i|>0$ for $i=1,2,3,4,5$, otherwise, $Q_{\infty}$ is a bipartite. Thus for any $i$, $|E'_i|$ and $|E'_{i+1}|$ cannot be both 0. Without loss of generality, we can assume $|E'_1|,|E'_3|,|E'_5|>0$ and $|E'_2|,|E'_4|\ge0$.

 So, the vertices in the five subset $V_i$ can be denoted as following:

$$V_1=\{v_{1,j};j=1,\ldots,|E'_1|+|E'_5|\},$$
$$V_2=\{v_{2,j};j=1,\ldots,|E'_1|+|E'_2|\},$$
$$V_3=\{v_{3,j};j=1,\ldots,|E'_2|+|E'_3|\},$$
$$V_4=\{v_{4,j};j=1,\ldots,|E'_3|+|E'_4|\},$$
$$V_5=\{v_{5,j};j=1,\ldots,|E'_4|+|E'_5|\},$$
note that $|E'_2|$ and $|E'_4|$ are possibly 0.

Now, we get the spanning path:

$$v_{1,1}v_{2,1}v_{1,2}v_{2,2}\cdots v_{1,|E'_1|}v_{2,|E'_1|}v_{3,1}$$
$$v_{2,|E'_1|+1}v_{3,2}\cdots v_{2,|E'_1|+|E'_2|}v_{3,|E'_2|+1}v_{4,1}v_{3,|E'_2|+2}v_{4,2}\cdots v_{3,|E'_3|+|E'_2|}$$
$$v_{4,|E'_3|}v_{5,1}v_{4,|E'_3|+1}v_{5,2}v_{4,|E'_3|+2}\cdots v_{4,|E'_3|+|E'_4|}$$
$$v_{5,|E'_4|+1}v_{1,|E'_1|+1}v_{5,|E'_4|+2}v_{1,|E'_1|+2}\cdots v_{5,|E'_4|+|E'_5|}v_{1,|E'_1|+|E'_5|}.$$

Note that if $|E'_2|=0$, the section $$v_{3,1}v_{2,|E_1'|+1}v_{3,2}\cdots v_{2,|E'_1|+|E'_2|}v_{3,|E'_2|+1}v_{4,1}$$ in the path is shortened to $v_{3,1}v_{4,1}$. And if $|E'_4|=0$ then the section $$v_{5,1}v_{4,|E'_3|+1}v_{5,2}v_{4,|E'_3|+2}\cdots v_{4,|E'_3|+|E'_4|}v_{5,|E'_4|+1}v_{1,|E'_1|+1}$$ in the path is shortened to $v_{5,1}v_{1,|E'_1|+1}$.
\end{enumerate}\end{proof}

\subsection{The proof of Theorem \ref{thm1}}
By Lemma \ref{lmDcoclique} and Theorem \ref{thm4}, we only need to prove that $Q_0$ is Hamiltonian.
The proof is divided into two parts. In the first part, we construct an auxiliary graph, a cycle, namely $\Gamma$. In the second part, with the help of $\Gamma$, we find a subgraph $H$ of $Q_0$,  which is the Hamiltonian cycle we want.
\subsubsection{Three trivial cases}
\begin{enumerate}
\item If $G$ is triangle-free, then $Q_0$ does not exist at all, then there is nothing to prove.
\item If there is only one clique, as defined in Lemma \ref{lm5}, with size at least 3, i.e. $Q_0=Q_1$, then obviously $Q_0$ is Hamiltonian, since it is just a clique.
\item If there are only two cliques, as defined in Lemma \ref{lm5}, with size at least 3, i.e. $V(Q_0)=V(Q_1)\cup V(Q_2)$, then choose an edge $v_1v_2$ from $Q_2$. We claim that $v_1$ and $v_2$ are adjacent to at least $|Q_1|-1$ vertices in $Q_1$. Otherwise, if there are two vertices $v_3$ and $v_4$ in $Q_1$ are adjacent neither $v_1$ nor $v_2$, then $v_1v_2$ and $v_3v_4$ form a $2K_2$.

Since $|Q_1|-1\ge2$, then we can choose two distinct vertices in $Q_1$, namely $v_5$ and $v_6$ such that $v_1v_5,~v_2v_6\in E(G)$. And we have a path in  $Q_1$, with $v_5$ and $v_6$ as endpoints and visiting each vertex in $Q_1$ exactly once, and also a path in $Q_2$, with $v_1$ and $v_2$ as endpoints and visiting each vertex in $Q_2$ exactly once.

Then these two paths and $v_1v_5$ and $v_2v_6$ form the Hamiltonian cycle we want.
\end{enumerate}

\subsubsection{The construction of $\Gamma$}

Now, we assume there are at least three cliques, as defined in Lemma \ref{lm5}, of size at least 3, i.e. $m\ge3$.

First, the vertex set of $\Gamma$, $V(\Gamma)=\{u_1,\ldots,u_m\}$, each $u_i$ corresponds to $Q_i$. 
Then we connect all these $u_i$ ($i=1,\ldots,m$) with a cycle, and we call this cycle $\Gamma$.




\subsubsection{The construction of $H$}
Now, with the help of $\Gamma$, we are going to find a Hamiltonian cycle in $Q_0$. 
The construction of $H$ is divided into two steps. In the first step, we find some {\bf green edges} in $Q_0$ corresponding to the edges in $\Gamma$. In the second step, we draw a path with {\bf yellow edges} in each $Q_i$, $i=1,\ldots,m$. Then the green edges and the yellow paths will make the Hamiltonian cycle in $Q_0$.

For the first step:

For any edge $u_iu_j$ $(1\le i<j\le m)$, we want to realize it into $H$ as a {\bf green edge}. Let us look at $Q_j$ first. We know that $u_j$ is incident to two edges in $\Gamma$, so there is only one vertex in $Q_j$, say $v_{j,1}$, is incident to another green edge. And $|Q_j|\ge3$, so there are at least two vertices $v_{j,2}$ and $v_{j,3}$ other than $v_{j,1}$ in $Q_j$. Both of $v_{j,2}$ and $v_{j,3}$ are dominated by $Q_i$. Then we claim that $v_{j,2}$ and $v_{j,3}$ have at least two neighbors in $Q_i$. Otherwise, if $v_{j,2}$ and $v_{j,3}$ have only one neighbor in $Q_i$, by $|Q_i|\ge3$, we know there are at least two vertices, say $v_{i,1}$ and $v_{i,2}$, who are adjacent to neither $v_{j,2}$ nor $v_{j,3}$. Then $v_{j,2}v_{j,3}$ and $v_{i,1}v_{i,2}$ form a $2K_2$.

Now we can assume $v_{j,2}v_1,~v_{j,3}v_2\in E(G)$, where $v_1,~v_2\in Q_i$. What is more, in $v_1$ and $v_2$, there is at most one of them, say $v_1$, is incident to another green edge, since there are only two edges incident to $u_i$ in $\Gamma$. Then we select $v_{j,3}v_2$ into $H$ as a {\bf green edge}.

From this process, we guarantee that each vertex in $Q_0$ is incident to at most one green edge in $H$.

For the second step:

Here we want to find a path, in each $Q_i$, $i=1,\dots,m$, of {\bf yellow edges}, after adding all the green edges.

For any $Q_i$, there are two distinct vertices incident to green edges, denoted as $v_{i,a}$ and $v_{i,b}$. Because $Q_i$ is a clique, so we can find a path in $Q_i$ with endpoints $v_{i,a}$ and $v_{i,b}$ and visiting each vertex in $Q_i$ exactly once. All the edges in this path are called {\bf yellow edges}.

Obviously, all the green edges and yellow edges together here make a Hamiltonian cycle in $Q_0$.  \qed

\section{The Proof of Theorem \ref{thm2}}

For an integer $p$, let $p*G$ denote the multigraph obtained from $G$ by taking each edge $p$ times. 
Thus a $k$-walk is a subgraph of $(2k)*G$. 


The $(\frac{1}{k-1}+\epsilon)$-toughness condition with $\epsilon>0$ 
is only used for the dominating cliques sequence defined in Lemma \ref{lm5} 
satisfying $m=1$ and $s\geq 1$. Namely, we prove the following
slightly stronger
\begin{theorem}\label{opt}
For any $2K_2$-free graph $G$, if $G$ is $(\frac{1}{k-1}+\epsilon)$-tough, 
$\epsilon>0$, then $G$ admits a $k$-walk. 

Moreover, a $\frac{1}{k-1}$-tough $G$ admits a $k$-walk if $G$ either has at
least two dominating cliques with size at least 3 in the dominating cliques
sequence defined in Lemma \ref{lm5}, i.e. $m\ge2$, or if $G$ is {\em split},
i.e. $m=1$, $s=0$.  
\end{theorem}




The following lemma is the key technique in the proof of Theorem \ref{thm2}.

\begin{lemma}\label{addtec}
For any graph $G$, if it allows a decomposition: $V(G)=V(Q)\cup V(D)$ such that:

\begin{enumerate} 
\item if the induced subgraphs $Q$ is Hamiltonian and $D$ is a coclique, and if $G$ is $\frac{1}{k-1}$-tough, then $G$ admits a $k$-walk.
\item if the induced subgraphs $Q$ admits a 2-walk, where only one vertex $v_{\infty}$ is visited twice in the walk and all other vertices are visited only once, $D$ is a coclique, and if $G$ is $(\frac{1}{k-1}+\epsilon)$-tough, then $G$ admits a $k$-walk.

\end{enumerate}
\end{lemma}

\begin{proof}
For part (1), choose any subset $D_0\subset D$, by the $\frac{1}{k-1}$-tough condition, $D_0$ has at least $\lceil\frac{|D_0|}{k-1}\rceil$ neighbors in $Q$. Then by Hall Theorem, there is a subset of edges $E'\subset E(G)$ such that each edge $e\in E'$ has one vertex in $D$ while another vertex in $Q$. Additionally, each vertex in $D$ is incident to exactly one edge in $E'$, while each vertex in $Q$ is incident to at most $k-1$ edges in $E'$.
Then these edges in $E'$ and the edges in the Hamilton cycle $H$ of $Q$ form a 2-walk of $G$.

For part (2), choose any subset $D_0\subset D$, by the $\frac{1}{k-1}$-tough condition, $D_0$ has at least $\lceil(\frac{1}{k-1}+\epsilon)|D_0|\rceil$ neighbors in $Q$. Now split each vertex in $Q$ into $k-1$ copies, the set of all these split vertices is denoted as $Q^k$ then $D_0$ has at least $(k-1)\lceil(\frac{1}{k-1}+\epsilon)|D_0|\rceil\ge|D_0|+1$ vertices, arbitrarily choose one copy of $v_{\infty}$, namely $v^*$, then $D_0$ has at least $|D_0|$ vertices in $Q^k-\{v^*\}$. Then there is a matching from $D$ to $Q^k-\{v^*\}$. Now collapse these copies back to $Q$, then each vertex in $D$ is incident to exactly one edges from the matching. Each vertex in $Q$ takes over at most $k-1$ edges from the matching, while $v_{\infty}$ takes at most $k-2$ edges from the matching. Then all these edges from the matching and the edges from the 2-walk in $Q$ form the k-walk of $G$.
\end{proof}


\subsection{Triangle-free case, i.e. $m=0$}
Now, we first look at the triangle-free case.

\begin{theorem}\label{trifree}
If $G$ is triangle-free, i.e. $\omega(G)=2$, then $(\frac{1}{k-1}+\epsilon)$-tough implies a $k$-walk.
\end{theorem}




\begin{proof}
As the definition before, $G$ has a weakly-dominating sequence $\{Q_1,\ldots,Q_s\}$, where $|Q_i|=2$.

{\bf Case 1.} If $s=1$, then obviously, $Q_1$ is dominating, since after deleting $Q_1$, there is only a set $D$ of isolated vertices.
By "triangle-free", each vertex $v\in D$ is adjacent to exactly one vertex in $Q_1$. By $(\frac{1}{k-1}+\epsilon)$-tough, each vertex in $Q_1$ is adjacent to at most $k-2$ vertices in $D$. Then we can draw the $k$-walk easily.

{\bf Case 2.} If $s=2$, then $V(G)=V(Q)\cup V(D)$, where $Q$ admits a path $u_1u_2u_3u_4$, while $D$ is a coclique.


If $u_1$ is adjacent to $u_4$, then $Q$ is hamiltonian, then by Lemma \ref{addtec}, there is a k-walk.

Now we assume $u_1$ is not adjacent to $u_4$.
By triangle-free, $u_1$ is not adjacent to $u_3$, and $u_2$ is not adjacent to $u_4$. Thus $Q$ is the path with no other edges.

If each vertex $v\in D$ is adjacent to exactly one vertex in $Q$, then by $(\frac{1}{k-1}+\epsilon)$-tough, $u_1$ (or $u_4$) is adjacent to at most $k-2$ vertices in $D$, while $u_2$ (or $u_3$) is adjacent to at most $k-3$ vertices in $D$. Then we can find the $k$-walk easily.

Now we assume there are some vertices in $D$ who are adjacent to at least two vertices in $Q$.
By triangle-free, they are not allowed to be adjacent to two successive vertices in $Q$.

\begin{enumerate}
\item assume $v_0\in D$ is adjacent to $u_1$ and $u_4$, then denote $Q'=Q\cup\{v_0\}$ and $D'=D-\{v_0\}$. Then $V(G)=V(Q')\cup V(D')$ while $Q'$ is hamiltonian and $D'$ is a coclique. Then by $(\frac{1}{k-1}+\epsilon)$-tough, we can find the $k$-walk easily.
\item assume $v_0\in D$ is adjacent to $u_1$ and $u_3$, ($u_2$ and $u_4$ case is similar), then denote $Q'=Q\cup\{v_0\}$ and $D'=D-\{v_0\}$. Then $V(G)=V(Q')\cup V(D')$ while $D'$ is a coclique. Now look at $Q'$, $Q'$ is a cycle $u_1u_2u_3v_0$ and a suspended edge $u_3u_4$, so $Q'$ is a 2-walk, where $u_1$, $u_2$, $u_4$, $v_0$ are visited once and $u_3$ is visited twice. Now we consider $u_3$ as the role of $v_{\infty}$ in Lemma \ref{addtec}, then we can find the $k$-walk with the assumption $(\frac{1}{k-1}+\epsilon)$-tough.
\end{enumerate}

{\bf Case 3.} If $s\ge3$, then $V(G)=V(Q)\cup V(D)$, and we require the sequence $\{Q_1,\ldots,Q_s\}$ has the maximum length possible. We know that $Q$ admits a path namely $u_1u_2\cdots u_{2s}$.

If both of $u_1$ and $u_{2s}$ have neighbors in $D$:
\begin{enumerate}
\item if $v_0\in D$ is adjancent to both $u_1$ and $u_{2s}$, then $Q'=Q\cup\{v_0\}$ is hamiltonian and $D'=D-\{v_0\}$ is a coclique. Then we can find the $k$-walk by Lemma \ref{addtec}.
\item if there are $v_1\neq v_2$ in $D$, where $v_1u_1,~v_2u_{2s}\in E(G)$, then we can find a longer sequence namely $Q'_1=v_1u_1$, $Q'_2=u_2u_3$,\ldots, $Q'_{s+1}=u_{2s}v_2$ in $G$. Contradition with the assumption.
\end{enumerate}

Now, we only consider the situation that at most one in $u_1$ and $u_{2s}$ has neighbors in $D$. Assume that $u_{2s}$ has {\bf no} neighbor in $D$.

Consider the $2K_2$-free condition with $u_1u_2$ and $u_{2s-1}u_{2s}$.
\begin{enumerate}
\item if $u_1u_{2s}\in E(G)$, then $Q$ is hamiltonian, then we can easily find the k-walk in $G$
\item if $u_1u_{2s-1}\in E(G)$, then $Q$ is 2-walk, with $u_{2s-1}$ playing the role of "$v_{\infty}$".
\item if $u_2u_{2s}\in E(G)$, then $Q$ is a 2-walk, with $u_2$ playing the role of "$v_{\infty}$".
\item if $u_2u_{2s-1}\in E(G)$, then denote $Q'=Q-\{u_{2s}\}$, $D'=D\cup \{u_{2s}\}$. Recall the assumption that $u_{2s}$ has no neighbor in $D$, so $D'$ is a coclique, and $Q'$ is the 2-walk, with $u_2$ playing the role of $v_{\infty}$. Then we can find the $k$-walk we want.
\end{enumerate}

\end{proof}


















\subsection{When $m\ge1$} 



We have already solved the case where $G$ is triangle-free in Theorem \ref{trifree}. So, we always assume $|Q_1|\ge3$ in the following proof.

By Lemma \ref{addtec}, we only need to proof the induced subgraph $Q$ on the vertex set $V(Q)=V(Q_0)\cup V(Q_{\infty})$ is Hamiltonian when $m\ge2$, (admits 2-walk, with only one vertex $Q_{\infty}$ visited at most twice, and all other vertices visited exactly once, when $m=1$.)

Let $G$ be a $(\frac{1}{k-1}+\epsilon)$-tough $2K_2$-free graph, for some $\epsilon>0$. The sequence of cliques defined in Lemma \ref{lm5} is $\{Q_1,\ldots,Q_m,\ldots,Q_{m+s}\}$, where $|Q_i|\ge3$, for $1\le i\le m$ and $|Q_i|=2$ for $m+1\le i\le m+s$. Additionally, $Q_i$ dominates $Q_j$ for $i\le m$ and $i<j$; $Q_i$ weakly dominates $Q_j$ for $m<i<j$. And $Q_{\infty}$ is the induced subgraph of $G$ on all the vertices of $Q_i$, $i>m$, as defined in Lemma \ref{lm6}. 

\subsubsection{When $s=0$}
When $s=0$, we have $Q=Q_0$, since $Q_{\infty}$ does not exist. Then by Theorem \ref{thm1}, $Q$ is Hamiltonian, and by Lemma \ref{addtec}, $G$ admits a $k$-walk, under the assmuption $G$ is $\frac{1}{k-1}$-tough.
 

\subsubsection{When $m\ge2$, $s\ge1$}
Denote a spanning path in $Q_{\infty}$ as $u_1\cdots u_{2s}$. Because each vertex in $Q_{\infty}$ is dominated by each clique $Q_i$, for $1\le i\le m$, we can find $v_{1,1}\in Q_1$ and $v_{m,1}\in Q_m$ such that $u_1v_{1,1},~u_{2s}v_{m,1}\in E(G)$. Now, let us find a spanning path in $Q_m$, starting with $v_{m,1}$. Obviously, such a path exists since $Q_m$ is a clique. Denote such a path as $v_{m,1}\cdots v_{m,|Q_m|}$. 

We know that $Q_m$ is dominated by $Q_{m-1}$, then there is a vertex, namely $v_{m-1,1}$ adjacnet to $v_{m,|Q_m|}$, again, we can find a spanning path in $Q_{m-1}$, namely $v_{m-1,1}\cdots v_{m-1,|Q_{m-1}|}$. 

Repeat such process, in $Q_2$, we find a vertex $v_{2,1}$ adjacent to $v_{3,|Q_3|}$. Because $|Q_2|\ge3$, then there are at least two other vertices, besides $v_{2,1}$ in $Q_2$, namely $v_{2,a}$ and $v_{2,b}$. In $Q_1$, there are at least two vertices, besides $v_{1,1}$, namely $v_{1,c}$ and $v_{1,d}$. By $2K_2$-free condition, we can assume that $v_{2,a}v_{1,c}\in E(G)$. Then we pick up a spanning $v_{2,1}\cdots v_{2,a}$ in $Q_2$, and a spanning path $v_{1,1}\cdots v_{1,c}$ in $Q_1$. Then all these paths are connected together into a Hamilton cycle in $Q$.


\subsubsection{$m=1$, $s\ge1$}

Again, denote the spanning path in $Q_{\infty}$ as $u_1\cdots u_{2s}$.
If $u_1$ and $u_{2s}$ have two different neighbors in $Q_1$, say $u_1v_1,~u_{2s}v_2\in E(G)$, where $v_1,~v_2\in Q_1$, then we can pick up a spanning path in $Q_1$ with $v_1$ and $v_2$ as two endpoints. Then connect the spanning path in $Q_{\infty}$ and the spanning path in $Q_1$ together, we get the Hamilton cycle in $G$.

If $u_1$ and $u_2$ are adjacent to only one vertex, say $v_{\infty}$ in $Q_1$, then we can find a Hamilton cycle in $Q_1$ and the two endpoints of the spanning path in $Q_{\infty}$ is connected to $v_{\infty}$ and come into a 2-walk, with $v_{\infty}$ visited twice and all other vertices visited once. Then by Lemma \ref{addtec}, we deduce the $k$-walk from the assumption that $G$ is $(\frac{1}{k-1}+\epsilon)$-tough.







\qed



At last we pose two conjectures to finish this paper:


\begin{conjecture}
Every $3/2$-tough $2K_2$-free graph on at least three vertices has a 2-trail, i.e. a 2-walk with each edge appearing in the walk at most once.
\end{conjecture}


\begin{conjecture}
Every 2-tough $2K_2$-free graph on at least three vertices is Hamiltonian.
\end{conjecture}
























\bibliography{reftough}

\bibliographystyle{plain}






\end{document}
