\documentclass{amsart}
\usepackage{amsthm}
\usepackage{enumerate}
\usepackage{xcolor}

\theoremstyle{definition}
\newtheorem{theorem}{Theorem}
\newtheorem{problem}{Problem}
\newtheorem{corollary}{Corollary}
\newtheorem{remark}{Remark}
\newtheorem{conjecture}{Conjecture}
\newtheorem{lemma}{Lemma}
\newtheorem{step}{Step}
\newtheorem{case}{Case}
\newtheorem{definition}{Definition}
\newcommand{\ncc}{c}

\begin{document}
\author{Gao Mou}
\address{School of Physical and Mathematical Sciences, Nanyang Technological University, Singapore} 
\email{gaom0002@e.ntu.edu.sg} 
\author{Dmitrii V. Pasechnik}
\address{Department of Computer Science, The University of Oxford, UK}
\email{dimpase@cs.ox.ac.uk}

\title{On $k$-walks in $2K_2$-free graphs}
\begin{abstract}
We show that every $\frac{1}{k-1}$-tough
$2K_2$-free graph admits a $k$-walk, and it can be found in polynomial time. 
For this class of graphs, this proves a
long-standing conjecture due to Jackson and Wormald (1990). The proof is based on a result by Veldman \cite{veldman83}.
Additionally, we also get some similar results of some other graphs.
\end{abstract}

\maketitle

\section{Introduction}
A graph $G$ is called $\beta$-{\em tough}, for a real $\beta>0$, if for any $p\geq 2$ it
cannot be split into $p$ components by removing less than $p\beta$ vertices.  
This concept, a measure of graph connectivity and ``resilience'' under vertex subsets removal,
was introduced in 1973 by V. Chv\'{a}tal 
while studying   Hamiltonicity of graphs. For a survey of results on graph toughness till 2006
cf. \cite{MR2221006}.

Let $p*G$ denote the multigraph obtained from $G$ by taking each edge $p$ times. 
A $k$-{\em walk} is a spanning subgraph $W$ of $(2k)*G$, such that each vertex of $W$ 
has even degree at most $2k$. % (the evenness says that $W$ is Eulerian). 
In particular a graph has a 1-walk if and only if it is $K_2$ (i.e. one edge) or Hamiltonian.
In 1990 B.~Jackson and N.~Wormald conjectured \cite{jackson1990k} that for any integer $k\ge2$ a
$\frac{1}{k-1}$-tough graph $G$ admits a $k$-walk.
{For a survey of results on walks in graphs till 2005 cf. \cite{kouider2005connected}.}

In this paper, we prove that this conjecture is true, under the
assumption that $G$ is  $2K_2$-free, that is $G$ does not contain an induced
copy of the disjoint union of two edges. 
\begin{theorem}\label{thm2} 
For any integer $k\ge2$, every
$\frac{1}{k-1}$-tough $2K_2$-free graph $G$
admits a $k$-walk.
Moreover, the latter can be found in time polynomial in $|V(G)|$.
\end{theorem}

If we let the toughness value $\frac{1}{k-1}$ increase to 
$\frac{1}{k-2}$ then
one does not need $2K_2$-freeness. Indeed, it is shown in 
\cite{jackson1990k} that
every $\frac{1}{k-2}$-tough graph has a $k$-walk. 
Clearly, if $G$ is
Hamiltonian, then $G$ is 1-tough.  More generally, 
If $G$ has a $k$-walk, then $G$ is $\frac{1}{k}$-tough \cite{jackson1990k}.
However, the converse is not true already for $k=1$.  

This more or less summarises the situation with $t$-tough graphs, $t\leq 1$.
On the $t>1$ side 
a famous conjecture of V.~Chv\'{a}tal \cite{chvatal1973tough} claims
that there exists a constant $\beta$ such that every
$\beta$-tough graph is Hamiltonian.  
Towards this, 
M.~Ellingham and X.~Zha \cite{ellingham2000toughness} proved that
every 4-tough graph has a 2-walk.

Recently, in \cite{broersma2014toughness}, H.~Broersma, V.~Patel and A.~Pyatkin proved that 
every 25-tough 2$K_2$-free graph on at least three vertices is Hamiltonian.
Our Theorem~\ref{thm2} was inspired by this result.  
However, our approach is technically quite different. 

To prove Theorem~\ref{thm2}, we first prove
\begin{theorem}\label{addgen1} 
For any $2K_2$-free graph $G$, if $G$ is not a tree, then $G$ admits an edge-dominating cycle. And this edge-dominating cycle can be found in polynomial time in $|V(G)|$.
\end{theorem}

The terminology we use is  mostly from \cite{bomu08}; we include the following for the sake of
clarity.  
A {\em clique} means a complete subgraph, a {\em coclique} means a
induced subgraph containing no edge. 
For any
$v\in V$, let $v^\perp:=v^{\perp_G}$ denote the union of $\{v\}$ and the set of
neighbours of $v$ in $G$.  More generally, for $A\subset V$, let
$A^\perp:=A^{\perp_G}$ denote the set of vertices {\em dominated} by $A$, i.e.
$A^\perp=\bigcup\limits_{y\in A}v^\perp$, and $A$ is said to be {\em
dominating} if $A^\perp=V$.
Moreover, if $A^{\perp}=V$ and the induced subgraph $V-A$ is a coclique, then we say $A$ is {\em edge-dominating}. 
 
It should also lead to no confusion when we talk
about  dominating induced subgraphs $H$ of $G$, in the sense of $V(H)$ being
dominating, respectively a subset of vertices (e.g. set of vertices of a
subgraph) being dominated by $H$, etc.

\section{Proof of Theorem \ref{addgen1}}



Because $G$ is not a tree, then it has a cycle, say $C=x_1x_2\cdots x_kx_1$, where $k\ge3$.
If $C$ is edge-dominating, then every thing is done. Now assume $C$ is not edge-dominating. Then, there must be some edge $v_1v_2$ (assume there are $t$ such edges), with neither $v_1$ nor $v_2$ is on $C$. 
Since $G$ is $2K_2$-free, $v_1$ and $v_2$ have at least two neighbors on $C$. Otherwise, if there is at most one vertex on $C$ adjacent to $v_1$ or $v_2$, then a $2K_2$ will appear.

Now, assume $x_1v_1\in E(G)$, 
\begin{enumerate}
\item if $x_2v_1\in E(G)$, then $C'=x_1v_1x_2x_3\cdots x_kx_1$ is a longer cycle,
\item if $x_2v_2\in E(G)$, then $x_1v_1v_2x_2x_3\cdots x_kx_1$ is a longer cycle, 
\item if $x_2v_1,x_2v_2\not\in E(G)$, then apply the $2K_2$-free property to $v_1v_2$ and $x_2x_3$, we get either $x_3v_1\in E(G)$ or $x_3v_2\in E(G)$.
\begin{enumerate}
\item if $x_3v_2\in E(G)$, then $C'=x_1v_1v_2x_3\cdots x_kx_1$ is a longer cycle,
\item if $x_3x_2\not\in E(G)$, then $x_3v_1\in E(G)$.
\begin{enumerate}
\item if $x_2$ is adjacent to no vertex outside $C$, then we use $C'=x_1v_1x_3\cdots x_kx_1$ to instead $C$. We know that $C$ and $C'$ have the same length, but $C'$ dominates all the edges who are dominated by $C$, and $C'$ also dominates $v_1v_2$, who is not dominated by $C$. So $t$ becomes smaller.
\item if $x_2$ is adjacent to some vertices outside $C$, say $z$, for example.

Recall that $x_2$ is adjacent to neither $v_1$ nor $v_2$, then $z$ must be adjacent to either $v_1$ or $v_2$. If $zv_1\in E(G)$, then $C'=x_1v_1zx_2x_3\cdots x_kx_1$ is a longer cycle.
If $zv_2\in E(G)$, then $C'=x_1v_1v_2zx_2x_3\cdots x_kx_1$ is a longer cycle.
\end{enumerate}

\end{enumerate}
\end{enumerate}

Repeat the process above. We know the length of a cycle in $G$ is limited, then the process will stop in finite steps. Finally, there will be no edge independent with the cycle from the last step. That is the edge-dominating cycle we want.

\qed

Theoretically, Theorem \ref{addgen1} can be considered as a corollary of the following result in \cite{veldman83}. However, Veldman used contraposition in his proof. Our proof is constructive, and thus provides a clear algorithm to find the edge-dominating cycle.

\begin{theorem}\label{veldman2}{\cite[Theorem 2]{veldman83}}
Let $G$ be a graph other than a tree, if for every pair of vertex-disjoint edges $e$ and $f$ of $G$.
$$d(e)+d(f)\ge|V(G)|-2,$$
then $G$ admits an edge-dominating cycle.
\end{theorem}

Here $d(e)$ stands for the number of edges incident with $e$.
\section{Proof of Theorem \ref{thm2}}





The following lemma is the key technique in the proof of Theorem \ref{thm2}.

\begin{lemma}\label{addtec}
Let $V(G)$ have a partition $V(G)=V(Q)\cup V(D)$, 
so that the induced subgraph $D$ is a coclique.
\begin{enumerate} 
\item If the induced subgraph $Q$ is Hamiltonian 
and $G$ is $\frac{1}{k-1}$-tough, 
then $G$ admits a $k$-walk.\label{itone}
\item If the induced subgraphs $Q$ admits a 2-walk, 
where only one vertex $v_{\infty}$ is visited twice in the walk 
and all other vertices are visited once, and $G$ is $(\frac{1}{k-1}+\epsilon)$-tough, then $G$ admits a $k$-walk.\label{ittwo}
\end{enumerate}
\end{lemma}

\begin{proof}
Let $D_0\subset D$.
For part (\ref{itone}), 
by $\frac{1}{k-1}$-toughness, 
$D_0$ has at least $\left\lceil\frac{|D_0|}{k-1}\right\rceil$ neighbours in $Q$. By Hall's Theorem, 
there is $E'\subset E(G)$ such that each $e\in E'$ has one vertex in $D$ and the other in $Q$,
and each vertex in $D$ is incident to exactly one edge in $E'$, while each vertex in $Q$ is incident to at most $k-1$ edges in $E'$.
Then these ({doubled}) edges in $E'$ and the edges in the Hamiltonian cycle $H$ of $Q$ form a $k$-walk in $G$.

For part (\ref{ittwo}), by $\left(\frac{1}{k-1}+\epsilon\right)$-toughness, 
$D_0$ has at least $\left\lceil(\frac{1}{k-1}+\epsilon)|D_0|\right\rceil$ neighbours in $Q$. 
Split each vertex in $Q$ into $k-1$ copies and denote the set of all the split vertices by $Q^{k-1}$. 
Then $D_0$ has at least $(k-1)\lceil(\frac{1}{k-1}+\epsilon)|D_0|\rceil\ge|D_0|+1$ neighbours in $Q^{k-1}$. 
Pick one copy of $v_{\infty}$, namely $v^*\in Q^{k-1}$. Then $D_0$ has at least $|D_0|$ neighbours 
in $Q^{k-1}-\{v^*\}$. Thus by Hall's Theorem there is a matching from $D$ to $Q^{k-1}-\{v^*\}$. 
Now collapse these copies back to $Q$. Each vertex in $D$ is incident to exactly one edge from the resulting matching $E'$. 
Each vertex in $Q-\{v_{\infty}\}$ takes at most $k-1$ edges from $E'$,
while $v_{\infty}$ takes at most $k-2$ edges from $E'$. 
Then $2*E'$ and the edges from the 2-walk in $Q$ form a $k$-walk in $G$.
\end{proof}


Combining Lemmas \ref{addtec}  and Theorem \ref{addgen1},
we obtain Theorem \ref{thm2}.



\medskip

Now, let us state two conjectures to close this section.
\begin{conjecture}
Every $3/2$-tough $2K_2$-free graph with at least 3 vertices has a 2-trail, i.e. a 2-walk with each edge appearing in the walk at most once.
\end{conjecture}


\begin{conjecture}
Every 2-tough $2K_2$-free graph with at least 3 vertices is Hamiltonian.
\end{conjecture}

\section{On 2-walks in some other graphs}
As a generalization of $2K_2$-free graphs, let us look at $3K_2$-free graphs, which are the graphs without three vertex-disjoint edges as a reduced subgraph.

In \cite{veldman83}, Veldman proved that:
\begin{lemma}\label{gen3k2de}{\cite[Corollary 3.2]{veldman83}}
Let $G$ be a 2-connected graph. If the degree sum of every three vertex-disjoint edges of $G$ is at least $|V(G)|-1$, then $G$ admits an edge-dominating cycle.
\end{lemma}

Combine Lemma \ref{gen3k2de} and Lemma \ref{addtec}, we get:

\begin{theorem}\label{2w3k2f}
If a $3K_2$-free graph $G$ is 1-tough, then $G$ admits a 2-walk.
\end{theorem}

\begin{proof}
Clearly, if $G$ is 1-tough, then $G$ is 2-connected. So, by Lemma \ref{gen3k2de}, $G$ has an edge-dominating cycle. Thus, by Lemma \ref{addtec}, $G$ admits a 2-walk.
\end{proof}

Also in \cite{veldman83}, Veldman proved:

\begin{lemma}\label{lemdelta}{\cite[Theorem C]{veldman83}}
Let $G$ be a 2-connected graph, if the minimal degree $\delta\ge\frac{|V(G)|+2}{3}$, then every longest cycle of $G$ is edge-dominating.
\end{lemma}

\begin{lemma}\label{2conxiaoe}{\cite[Corollary 3.3.1]{veldman83}}
If $G$ is a 2-connected graph with $$|E(G)|\ge(\frac{(|V(G)|-4)(|V(G)|-5)}{2}+11),$$ then $G$ admits an edge-dominating cycle.
\end{lemma}

Obviously, combine Lemma \ref{addtec} with the two lemmas above, we get:

\begin{theorem}\label{mdeg2w}
For any 1-tough graph $G$, if the minimal degree $\delta\ge\frac{|V(G)|+2}{3}$, then $G$ admits a 2-walk.
\end{theorem}


\begin{theorem}
For any 1-tough graph $G$, if $$|E(G)|\ge(\frac{(|V(G)|-4)(|V(G)|-5)}{2}+11),$$ then $G$ admits a 2-walk.
\end{theorem}



















\subsection*{Acknowledgements.}
The authors thank Nick Gravin
for helpful comments on a draft of this text.
Research supported by Singapore MOE Tier 2 Grant MOE2011-T2-1-090 (ARC 19/11). 

\bibliography{reftough}
\bibliographystyle{abbrv}
%\bibliographystyle{plain}
\end{document}
